\documentclass[journal]{IEEEtran}
% If IEEEtran.cls has not been installed into the LaTeX system files,
% manually specify the path to it like:
% \documentclass[journal]{../sty/IEEEtran}

% Import setup file with includes
\input{setup.tex}

\title{RBE 550 Homework 1 Documentation}

% TODO: Fix authors
\author{Alex Tacescu | Spring 2021
% <-this % stops a space
% \thanks{M. Shell was with the Department
% of Electrical and Computer Engineering, Georgia Institute of Technology, Atlanta,
% GA, 30332 USA e-mail: (see http://www.michaelshell.org/contact.html).}% <-this % stops a space
% \thanks{J. Doe and J. Doe are with Anonymous University.}% <-this % stops a space
% \thanks{Manuscript received April 19, 2005; revised August 26, 2015.}
}

\bibliography{ref}

\begin{document}
    % \markboth{Journal of \LaTeX\ Class Files,~Vol.~14, No.~8, August~2015}%
    % {Shell \MakeLowercase{\textit{et al.}}: Bare Demo of IEEEtran.cls for IEEE Journals}

    % make the title area
    \maketitle

    \section{Introduction}
    This document accompanies my homework 1 submission. It is designed to explain my thought process when developing my code. It will also explain the 4 algorithms implemented: Breath First Search (BFS), Depth First Search (DFS), Dijkstra's Algorithm, and A*.

    \section{General Code Setup}
    The code is split up into two main sections, with 2 classes developed to accompany the search algorithms while also simplifying the code. The \lstinline{map2d} class (\lstinline{map2d.py}) is designed to store the map and add features such as returning local neighbors, getting distances between points, and getting occupancy data of nodes. It allows the main code in search.py to be significantly more brief and easier to read, while avoiding duplicate code. Similarly, the \lstinline{PriorityQueue} class (\lstinline{priorityqueue.py}) was created to keep a priority queue style data structure. It is important to note that lists were used instead of \lstinline{numpy} arrays to minimize dependencies, especially when running on a professor/TA computer.

    Testing the algorithms can be separated in to two sections. First of all, the search algorithms were developed using a basic map found in \lstinline{test_map.csv}. This allowed me to hone in the software and ensure that everything worked as planned. Afterwards, 3 additional maps were created (other than \lstinline{map.csv}) in order to test certain expected qualities of each algorithm. For example, a map was created to show the difference between BFS and DFS. The graphs shown in this paper will use the \lstinline{map.csv} default for familiarity. The graphing tool from the next subsection also was used to ensure all algorithms were performing as expected.
    
    \subsection{Additional Tool: Graphing Visited Nodes and Their Weights}
    In order to better debug and visualize the weights and priorities while implementing Dijkstra's algorithm (Section \ref{Dijkstra}) and A* (Section \ref{A*}), a graph was developed that displays the visited nodes and their weights. This was added to the map2d class to be easily used between algorithms. You can see an example in Figure \ref{fig:Dijkstra_Weights_Example}

    \begin{figure}[ht]
        \includegraphics[width=\linewidth]{figures/Dijkstra_Weights.png}
        \label{fig:Dijkstra_Weights_Example}
        \caption{Example of a map populated with the heuristic weights from Dijkstra's Algorithm}
    \end{figure} 

    \section{Breath First Search (BFS)} \label{BFS}
    The main concept behind BFS is that the algorithm explores a graph or map equally in all directions. It is extremely simple and useful, and usually leads to a complete search and an optimal path. However, it takes a long time to compute since there is no heuristic pushing the algorithm to look toward the goal. It is also important to note that BFS algorithms assume there is no difference moving between points on a graph/map - a major limitation especially when path planning.

    \subsection{Implementation}
    BFS is one of the most simple algorithms, usually taking only a few lines of code to fully implement. Below is some basic Python pseudocode describing my implementation of BFS: \cite{RedBlobGames_a*}

    \begin{lstlisting}[language=Python]
frontier = list()
frontier.append(start_node)
visited = dictionary()
visited[start_node] = None

while frontier not empty:
    current = frontier.get_first()
    if goal found in visited:
        exit loop and return path
    for neighbor in neighbors of current:
        if neighbor exists and \
            not in visited and \
            not obstacle:
            frontier.add_to_end(neighbor)
            visited[neighbor] = current
    \end{lstlisting}

    The output of the code above only gives a list of visited nodes and where they were visited from. Therefore, an additional section of code was needed to generate a path:

    \begin{lstlisting} [language=Python]
path = list()
curr_point = goal_node
while curr_point is not start:
        path.append(curr_point)
        curr_point = visited.get(curr_point)
path.append(start)
path.reverse()
    \end{lstlisting}

    Finally, to get the number of steps taken, I simply took the length of the \lstinline{visited} dictionary.

    \subsection{Results}

    \begin{figure}[ht]
        \includegraphics[width=\linewidth]{figures/BFS_LargeMap_Path.png}
        \label{fig:Dijkstra_Weights_Example}
        \caption{Breath First Search Results for a 10x10 map. The algorithm took 64 steps to find the solution.}
    \end{figure} 

    \begin{figure}[ht]
        \includegraphics[width=\linewidth]{figures/BFS_SmallMap_Path.png}
        \label{fig:Dijkstra_Weights_Example}
        \caption{Breath First Search result for a 4x4 map. The algorithm took 10 steps to find the solution}
    \end{figure} 

    \section{Depth First Search (DFS)} \label{DFS}
    DFS can be considered the antithesis of BFS; instead of spreading out uniformly, the algorithm picks a path and goes down it until a dead end. This often leads to very quick but non-optimal results. Like BFS, this search algorithm is blind to movement costs.

    \subsection{Implementation}
    DFS is very similar to BFS, but with a few minor modifications. Instead of appending new neighbors to the end of the frontier list, they are appended to the beginning to be considered immediately. Below is some basic Python pseudocode describing my implementation of DFS:

    \begin{lstlisting}[language=Python]
frontier = list()
frontier.append(start_node)
visited = dictionary()
visited[start_node] = None

while frontier not empty:
    current = frontier.get_first()
    if goal found in visited:
        exit loop and return path
    for neighbor in neighbors of current:
        if neighbor exists and \
            not in visited and \
            not obstacle:
            frontier.add_to_beginning(neighbor)
            visited[neighbor] = current
    \end{lstlisting}
        
    The output of the code above only gives a list of visited nodes and where they were visited from. Therefore, a code block needs to be added to generate a path (see DFS)

    Finally, to get the number of steps taken, I simply took the length of the \lstinline{visited} dictionary.

    \subsection{Results}

    \begin{figure}[ht]
        \includegraphics[width=\linewidth]{figures/DFS_LargeMap_Path.png}
        \label{fig:Dijkstra_Weights_Example}
        \caption{Depth First Search result for a 10x10 map. The algorithm took 33 steps to find the solution}
    \end{figure} 

    \begin{figure}[ht]
        \includegraphics[width=\linewidth]{figures/DFS_SmallMap_Path.png}
        \label{fig:Dijkstra_Weights_Example}
        \caption{Depth First Search result for a 4x4 map. The algorithm took 9 steps to find the solution}
    \end{figure} 
    
    \newpage

    \section{Dijkstra's Algorithm} \label{Dijkstra}
    Dijkstra's algorithm extends the usefulness Breath First Search by considering path heuristics. This is extremely useful to get a complete and optimal path from a start node to a goal node considering the movement effort going from one node to another. However, like BFS, it takes a very long time since there is no heuristic to push the algorithm to the goal node faster than other methods.

    \subsection{Implementation}
    Like stated above, Dijkstra's algorithm adds movement weight to a BFS-like algorithm. Therefore, weights need to be added to the visited and frontier data structures and need to be considered when moving from one to another

    \begin{lstlisting}[language=Python]
frontier = PriorityQueue()
frontier.put(start_node, 0) # 0 is the weight
came_from = dictionary()
came_from[start_node] = {
    'from': None,
    'cost': 0
}

while frontier not empty:
    (current_cost, current) = frontier.get_best()
    if goal found in came_from:
        exit loop and return path
    for neighbor in neighbors of current:
        neighbor_cost = current_cost + manhattan_distance(current, neighbor)
        if neighbor exists and \
            not in visited and \
            not obstacle:
            frontier.add(neighbor, neighbor_cost)
            came_from[neighbor] = {
                'from': current,
                'cost': neighbor_cost
            }
    \end{lstlisting}
        
    The output of the code above only gives a list of \lstinline{came_from} nodes and where they were visited from. Therefore, a code block needs to be added to generate a path:

    \begin{lstlisting} [language=Python]
path = list()
curr_point = goal_node
while curr_point is not start:
        path.append(curr_point)
        curr_point = came_from.get(curr_point).get('from')
path.append(start)
path.reverse()
    \end{lstlisting}

    Finally, to get the number of steps taken, I simply took the length of the \lstinline{came_from} dictionary.
    \subsection{Results}

    \begin{figure}[ht]
        \includegraphics[width=\linewidth]{figures/Dijkstra_LargeMap_Path.png}
        \label{fig:Dijkstra_Weights_Example}
        \caption{Dijkstra's Algorithm search result for a 10x10 map. The algorithm took 64 steps to find the solution}
    \end{figure} 

    \begin{figure}[ht]
        \includegraphics[width=\linewidth]{figures/Dijkstra_SmallMap_Path.png}
        \label{fig:Dijkstra_Weights_Example}
        \caption{Dijkstra's Algorithm search result for a 4x4 map. The algorithm took 10 steps to find the solution}
    \end{figure} 

    \section{A*} \label{A*}
    The A* algorithm is the best of all worlds when it comes to the 4 algorithms. It adds to Dijkstra's algorithm by adding an additional heuristic for estimated cost from the current node to the goal node. This drastically increases the speed of computation by skipping the breath first search-like exploration in all directions. However, the quality of this path is heavily dependent on the accuracy of the cost-to-goal heuristic. In some severe circumstances, the path outputted may not be optimal and the search will not be complete. This is usually due to a bad future path heuristic function.

    \subsection{Implementation}
    Like stated before, the major difference between the implementation of A* and Dijkstra's is in the cost function: an added 'estimated' cost to the goal. The cost-to heuristic is added to the future-cost heuristic to get a final weight. 

    \begin{lstlisting}[language=Python]
frontier = PriorityQueue()
frontier.put(start_node, 0) # 0 is the weight
came_from = dictionary()
came_from[start_node] = {
    'from': None,
    'cost': 0
}

while frontier not empty:
    (current_cost, current) = frontier.get_best()
    if goal found in came_from:
        exit loop and return path
    for neighbor in neighbors of current:
        neighbor_cost = current_cost - manhattan_distance(current, goal) + manhattan_distance(current, neighbor) + manhattan_distance(current, neighbor)
        if neighbor exists and \
            not in visited and \
            not obstacle:
            frontier.add(neighbor, neighbor_cost)
            came_from[neighbor] = {
                'from': current,
                'cost': neighbor_cost
            }
    \end{lstlisting}
        
    The output of the code above only gives a list of \lstinline{came_from} nodes and where they were visited from. Therefore, a code block needs to be added to generate a path:

    \begin{lstlisting} [language=Python]
        path = list()
        curr_point = goal_node
        while curr_point is not start:
                path.append(curr_point)
                curr_point = came_from.get(curr_point).get('from')
        path.append(start)
        path.reverse()
            \end{lstlisting}

    Finally, to get the number of steps taken, I simply took the length of the \lstinline{came_from} dictionary.
    \subsection{Results}

    \begin{figure}[ht]
        \includegraphics[width=\linewidth]{figures/A_LargeMap_Path.png}
        \label{fig:Dijkstra_Weights_Example}
        \caption{A* search result for a 10x10 map. The algorithm took 48 steps to find the solution}
    \end{figure} 

    \begin{figure}[ht]
        \includegraphics[width=\linewidth]{figures/A_SmallMap_Path.png}
        \label{fig:Dijkstra_Weights_Example}
        \caption{A* search result for a 4x4 map. The algorithm took 8 steps to find the solution}
    \end{figure} 
    \newpage
    \printbibliography

\end{document}